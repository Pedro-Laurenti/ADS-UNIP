%% abtex2-modelo-trabalho-academico.tex, v-1.9.7 laurocesar
%% Copyright 2012-2018 by abnTeX2 group at http://www.abntex.net.br/ 
%%
%% This work may be distributed and/or modified under the
%% conditions of the LaTeX Project Public License, either version 1.3
%% of this license or (at your option) any later version.
%% The latest version of this license is in
%%   http://www.latex-project.org/lppl.txt
%% and version 1.3 or later is part of all distributions of LaTeX
%% version 2005/12/01 or later.
%%
%% This work has the LPPL maintenance status `maintained'.
%% 
%% The Current Maintainer of this work is the abnTeX2 team, led
%% by Lauro César Araujo. Further information are available on 
%% http://www.abntex.net.br/
%%
%% This work consists of the files abntex2-modelo-trabalho-academico.tex,
%% abntex2-modelo-include-comandos and abntex2-modelo-references.bib
%%

% ------------------------------------------------------------------------
% ------------------------------------------------------------------------
% abnTeX2: Modelo de Trabalho Academico (tese de doutorado, dissertacao de
% mestrado e trabalhos monograficos em geral) em conformidade com 
% ABNT NBR 14724:2011: Informacao e documentacao - Trabalhos academicos -
% Apresentacao
% ------------------------------------------------------------------------
% ------------------------------------------------------------------------

\documentclass[
	% -- opções da classe memoir --
	12pt,				% tamanho da fonte
	openright,			% capítulos começam em pág ímpar (insere página vazia caso preciso)
	twoside,			% para impressão em recto e verso. Oposto a oneside
	a4paper,			% tamanho do papel. 
	% -- opções da classe abntex2 --
	%chapter=TITLE,		% títulos de capítulos convertidos em letras maiúsculas
	%section=TITLE,		% títulos de seções convertidos em letras maiúsculas
	%subsection=TITLE,	% títulos de subseções convertidos em letras maiúsculas
	%subsubsection=TITLE,% títulos de subsubseções convertidos em letras maiúsculas
	% -- opções do pacote babel --
	english,			% idioma adicional para hifenização
	brazil				% o último idioma é o principal do documento
	]{abntex2}

% ---
% Pacotes básicos 
% ---
\usepackage{lmodern}			% Usa a fonte Latin Modern			
\usepackage[T1]{fontenc}		% Selecao de codigos de fonte.
\usepackage[utf8]{inputenc}		% Codificacao do documento (conversão automática dos acentos)
\usepackage{indentfirst}		% Indenta o primeiro parágrafo de cada seção.
\usepackage{color}				% Controle das cores
\usepackage{graphicx}			% Inclusão de gráficos
\usepackage{microtype} 			% para melhorias de justificação
\usepackage{adjustbox}          % Quadros justificados

% ---
		
% ---
% Pacotes adicionais, usados apenas no âmbito do Modelo Canônico do abnteX2
% ---
\usepackage{lipsum}				% para geração de dummy text
% ---

% ---
% Pacotes de citações
% ---
\usepackage[brazilian,hyperpageref]{backref}	 % Paginas com as citações na bibl
\usepackage[alf]{abntex2cite}	% Citações padrão ABNT

% --- 
% CONFIGURAÇÕES DE PACOTES
% --- 

% ---
% Configurações do pacote backref
% Usado sem a opção hyperpageref de backref
\renewcommand{\backrefpagesname}{Citado na(s) página(s):~}
% Texto padrão antes do número das páginas
\renewcommand{\backref}{}
% Define os textos da citação
\renewcommand*{\backrefalt}[4]{
	\ifcase #1 %
		Nenhuma citação no texto.%
	\or
		Citado na página #2.%
	\else
		Citado #1 vezes nas páginas #2.%
	\fi}%
% ---

% ---
% Informações de dados para CAPA e FOLHA DE ROSTO
% ---
\titulo{PIM VI \\ Projeto Integrado Multidiciplinar VI}
\autor{Pedro Laurenti, Lucas Andrade e Allan Cândido}
\local{Brasil}
\data{Setembro de 2023}
\orientador{Robson Batista Alves}
\coorientador{Tarcisio Peres}
\instituicao{%
  Universidade Paulista - UNIP
  \par
  Curso de Análise e Desenvolvimento de Sistemas
  \par}
\tipotrabalho{Trabalho Científico}
% O preambulo deve conter o tipo do trabalho, o objetivo, 
% o nome da instituição e a área de concentração 
\preambulo{Trabalho científico redigido colocando em prática as habilidades e conhecimento adquiridos no terceiro período do curso.}
% ---


% ---
% Configurações de aparência do PDF final

% alterando o aspecto da cor azul
\definecolor{blue}{RGB}{41,5,195}

% informações do PDF
\makeatletter
\hypersetup{
     	%pagebackref=true,
		pdftitle={\@title}, 
		pdfauthor={\@author},
    	pdfsubject={\imprimirpreambulo},
	    pdfcreator={Pedro Laurenti e Lucas Andrade},
		pdfkeywords={Desenvolvimento}{Análise e Desenvolvimento}{Sistemas}{PIM VII}{UNIP}, 
		colorlinks=true,       		% false: boxed links; true: colored links
    	linkcolor=blue,          	% color of internal links
    	citecolor=blue,        		% color of links to bibliography
    	filecolor=magenta,      		% color of file links
		urlcolor=blue,
		bookmarksdepth=4
}
\makeatother
% --- 

% ---
% Posiciona figuras e tabelas no topo da página quando adicionadas sozinhas
% em um página em branco. Ver https://github.com/abntex/abntex2/issues/170
\makeatletter
\setlength{\@fptop}{5pt} % Set distance from top of page to first float
\makeatother
% ---

% ---
% Possibilita criação de Quadros e Lista de quadros.
% Ver https://github.com/abntex/abntex2/issues/176
%
\newcommand{\quadroname}{Quadro}
\newcommand{\listofquadrosname}{Lista de quadros}

\newfloat[chapter]{quadro}{loq}{\quadroname}
\newlistof{listofquadros}{loq}{\listofquadrosname}
\newlistentry{quadro}{loq}{0}

% configurações para atender às regras da ABNT
\setfloatadjustment{quadro}{\centering}
\counterwithout{quadro}{chapter}
\renewcommand{\cftquadroname}{\quadroname\space} 
\renewcommand*{\cftquadroaftersnum}{\hfill--\hfill}

\setfloatlocations{quadro}{hbtp} % Ver https://github.com/abntex/abntex2/issues/176
% ---

% --- 
% Espaçamentos entre linhas e parágrafos 
% --- 

% O tamanho do parágrafo é dado por:
\setlength{\parindent}{1.3cm}

% Controle do espaçamento entre um parágrafo e outro:
\setlength{\parskip}{0.2cm}  % tente também \onelineskip

% ---
% compila o indice
% ---
\makeindex
% ---

% ----
% Início do documento
% ----
\begin{document}

% Seleciona o idioma do documento (conforme pacotes do babel)
%\selectlanguage{english}
\selectlanguage{brazil}

% Retira espaço extra obsoleto entre as frases.
\frenchspacing 

% ----------------------------------------------------------
% ELEMENTOS PRÉ-TEXTUAIS
% ----------------------------------------------------------
% \pretextual

% ---
% Capa
% ---
\imprimircapa
% ---

% ---
% Folha de rosto
% (o * indica que haverá a ficha bibliográfica)
% ---
\imprimirfolhaderosto*
% ---

% ---
% Inserir a ficha bibliografica
% ---

% Isto é um exemplo de Ficha Catalográfica, ou ``Dados internacionais de
% catalogação-na-publicação''. Você pode utilizar este modelo como referência. 
% Porém, provavelmente a biblioteca da sua universidade lhe fornecerá um PDF
% com a ficha catalográfica definitiva após a defesa do trabalho. Quando estiver
% com o documento, salve-o como PDF no diretório do seu projeto e substitua todo
% o conteúdo de implementação deste arquivo pelo comando abaixo:
%
% \begin{fichacatalografica}
%     \includepdf{fig_ficha_catalografica.pdf}
% \end{fichacatalografica}

\begin{fichacatalografica}
	\sffamily
	\vspace*{\fill}					% Posição vertical
	\begin{center}					% Minipage Centralizado
	\fbox{\begin{minipage}[c][8cm]{13.5cm}		% Largura
	\small
	\imprimirautor
	%Sobrenome, Nome do autor
	
	\hspace{0.5cm} PIM V Projeto Integrado Multidiciplinar V - 
	\imprimirlocal, \imprimirdata \ --
	
	\hspace{0.5cm} \thelastpage p.\\
	
	\hspace{0.5cm} \imprimirorientadorRotulo~\imprimirorientador\\
	
	\hspace{0.5cm}
	\parbox[t]{\textwidth}{\imprimirtipotrabalho~--~\imprimirinstituicao,
	\imprimirdata.}\\
	
	\hspace{0.5cm}
		1. Eccomerce.
		2. Desenvolvimento.
		3. Análise de Sistemas.
		I. Orientador.
		II. UNIP - Universidade Paulista.
		III. Faculdade de  Análise e Desenvolvimento de Sistemas.
		IV. PIM VII - Projeto Integrado Multidiciplinar.  		
	\end{minipage}}
	\end{center}
\end{fichacatalografica}
% ---

% ---
% Inserir folha de aprovação
% ---

% Isto é um exemplo de Folha de aprovação, elemento obrigatório da NBR
% 14724/2011 (seção 4.2.1.3). Você pode utilizar este modelo até a aprovação
% do trabalho. Após isso, substitua todo o conteúdo deste arquivo por uma
% imagem da página assinada pela banca com o comando abaixo:
%
% \begin{folhadeaprovacao}
% \includepdf{folhadeaprovacao_final.pdf}
% \end{folhadeaprovacao}
%
\begin{folhadeaprovacao}

  \begin{center}
    {\ABNTEXchapterfont\large\imprimirautor}

    \vspace*{\fill}\vspace*{\fill}
    \begin{center}
      \ABNTEXchapterfont\bfseries\Large\imprimirtitulo
    \end{center}
    \vspace*{\fill}
    
    \hspace{.45\textwidth}
    \begin{minipage}{.5\textwidth}
        \imprimirpreambulo
    \end{minipage}%
    \vspace*{\fill}
   \end{center}
        
   Trabalho aprovado. \imprimirlocal, 24 de novembro de 2012:

   \assinatura{\textbf{\imprimirorientador} \\ Orientador} 
   \assinatura{\textbf{Professor} \\ Convidado 1}
   \assinatura{\textbf{Professor} \\ Convidado 2}
   %\assinatura{\textbf{Professor} \\ Convidado 3}
   %\assinatura{\textbf{Professor} \\ Convidado 4}
      
   \begin{center}
    \vspace*{0.5cm}
    {\large\imprimirlocal}
    \par
    {\large\imprimirdata}
    \vspace*{1cm}
  \end{center}
  
\end{folhadeaprovacao}
% ---

% ---
% Dedicatória
% ---
\begin{dedicatoria}
	\vspace*{\fill}
	\centering
	\noindent
	\textit{Este projeto é dedicado a todos os desenvolvedores que já falharam várias vezes, mas nunca desistiram de suas paixões e ideias.}

	\begin{quote}
	\textit{``Ser feliz ao realizar a jornada pode ser muito melhor do que chegar ao destino com sucesso.'' - Jordan Peterson.}
	\end{quote}

	\vspace*{\fill}

\end{dedicatoria}
% ---

% ---
% Agradecimentos
% ---
\begin{agradecimentos}
	Os agradecimentos principais são direcionados à todos aqueles que contribuíram para que a produção deste trabalho acadêmico.

	Agradecimentos especiais aos desenvolvedores do \abnTeX e ao professor Miguel Frasson - pelas orientações.

\end{agradecimentos}
% ---


% ---
% Epígrafe
% ---
\begin{epigrafe}
	\vspace*{\fill}
	\begin{flushright}
		\textit{``A máquina moderna é um instrumento de poder sem precedentes; e sua falha é que não há precedentes que possam nos ensinar como lidar com isso." - (G.K. Chesterton)}
	\end{flushright}
\end{epigrafe}
% ---

% ---
% RESUMOS
% ---

% resumo em português
\setlength{\absparsep}{18pt} % ajusta o espaçamento dos parágrafos do resumo
\begin{resumo}
	Este é um Projeto Integrado Multidisciplinar (PIM) abrangendo diversas áreas de conhecimento, incluindo Projeto de Sistemas Orientada a Objetos, Empreendedorismo, Gestão de Qualidade e Gerenciamento de Projeto de Software, com o objetivo de realizar um levantamento completo e uma análise detalhada dos requisitos para um Sistema de Marketplace para compra e venda de produtos diversos via app/web. Em um contexto de constante avanço tecnológico, não é possível para o comércio feito de forma tradicional vencer todos os desafios para fornecer a melhor experiência de compra ao consumidor. A adoção de marketplace se faz necessária, tal sistema moderniza os processos e provê mais facilidade e variedade nos processos de compra para ambos consumidor e loja. Para atender aos requisitos desse projeto, é fundamental realizar uma análise aprofundada das necessidades e desafios de mercado através de um plano de negócios além de ser fundamental incluir a organização dos processos e ações possíveis de serem tomadas pelas partes atuantes no sistema, alterações, consultas e exclusões, bem como o estabelecimento de funcionalidades que permitam o uso eficiente do sistema por atendentes, estoquistas e o supervisor da loja, com a ajuda de diagramas. Ao considerar a interdisciplinaridade desse projeto, os conhecimentos adquiridos nas disciplinas de Projeto de Sistemas Orientada a Objetos, Empreendedorismo, Gestão de Qualidade e Gerenciamento de Projeto de Software se tornam essenciais para garantir a eficiência e a integridade do sistema desenvolvido. Dessa forma, espera-se apresentar um software prático, confiável e adaptado às necessidades específicas da loja, contribuindo para a otimização das operações de vendas.

 \textbf{Palavras-chave}: projeto de sistemas, empreendedorismo, interdisciplinaridade, software.
\end{resumo}

% resumo em inglês
\begin{resumo}[Abstract]
 \begin{otherlanguage*}{english}
	This project aims to develop an Integrated Multidisciplinary Project (IMP) encompassing various areas of knowledge, including Object-Oriented Systems Project, Entrepreneurship, Quality Management and Software Project Management with the objective of conducting a comprehensive survey and detailed analysis of the requirements for a Marketplace System for buying and selling various products via app/web. In a context of constant technological advancements, it is not possible for traditional commerce to overcome all challenges to provide the best consumer shopping experience. The adoption of a marketplace is necessary, said system modernizes processes and provides greater ease and variety in purchasing processes for both consumers and stores. To meet the requirements of this project, it is essential to conduct a detailed survey and in-depth analysis of market needs and challenges through a business plan, in addition to being essential to include the organization of processes and possible actions to be taken by the parties working in the system, changes, consultations and deletions, as well as the establishment of functionalities that allow efficient use of the system by attendants, stockists and the store supervisor, with the help of diagrams. Considering the interdisciplinary nature of this project, the knowledge acquired in the disciplines of Object-Oriented Systems Project, Entrepreneurship, Quality Management and Software Project Management becomes essential to ensure the efficiency and integrity of the developed system. Thus, the goal is to present a practical, reliable software tailored to the specific needs of the store, contributing to the optimization of sales operations

   \vspace{\onelineskip}
 
   \noindent 
   \textbf{Keywords}: systems project, entrepreneurship, interdisciplinary approach, software.
 \end{otherlanguage*}
\end{resumo}

% ---


% ---
% inserir o sumario
% ---
\pdfbookmark[0]{\contentsname}{toc}
\tableofcontents*
\cleardoublepage
% ---


% ---
% inserir lista de ilustrações
% ---
\pdfbookmark[0]{\listfigurename}{lof}
\listoffigures*
\cleardoublepage
% ---

% ---
% inserir lista de quadros
% ---
\pdfbookmark[0]{\listofquadrosname}{loq}
\listofquadros*
\cleardoublepage
% ---

% ---
% inserir lista de tabelas
% ---
\pdfbookmark[0]{\listtablename}{lot}
\listoftables*
\cleardoublepage
% ---

% ---
% inserir lista de abreviaturas e siglas
% ---
\begin{siglas}
  \item[ABNT] Associação Brasileira de Normas Técnicas
  \item[abnTeX] ABsurdas Normas para TeX
  \item[SEO] Search Engine Optimization
  \item[C2C] Client to client
  \item[IA] Inteligência Artificial
  \item[RNF] Requisito Não funcional
  \item[RF] Requisito Funcional
\end{siglas}
% ---

% ----------------------------------------------------------
% ELEMENTOS TEXTUAIS
% ----------------------------------------------------------
\textual

% ----------------------------------------------------------
% Introdução (exemplo de capítulo sem numeração, mas presente no Sumário)
% ----------------------------------------------------------
\chapter{Introdução}
% ----------------------------------------------------------

A adoção de sistemas de marketplace para facilitar a compra e venda de uma ampla variedade de produtos por meio de aplicativos e sites tornou-se fundamental na atual era tecnológica. Essas plataformas desempenham um papel importante ao reunir compradores e vendedores em um único ambiente virtual, oferecendo inúmeras vantagens para ambos os lados.

Para os consumidores, a importância desses sistemas reside na comodidade e praticidade que oferecem. Eles permitem que os clientes naveguem por uma ampla gama de produtos, comparem preços, leiam avaliações de outros compradores e efetuem compras com facilidade, tudo a partir de um dispositivo conectado à internet. Isso economiza tempo e esforço, tornando a experiência de compra mais eficiente e satisfatória.

Já para os comerciantes, os sistemas de marketplace representam uma oportunidade de alcançar um público muito mais amplo do que seria possível em suas operações tradicionais. Eles podem aproveitar a infraestrutura tecnológica e a base de clientes já existente na plataforma, economizando em custos de marketing e expandindo seus horizontes de negócios além de simplificar a administração de pedidos, estoque e pagamentos, tudo isso proporciona uma operação mais eficiente e escalável.

Além disso, a confiança é um fator crítico em qualquer transação comercial, e os sistemas de marketplace geralmente oferecem uma camada adicional de segurança. Muitos deles implementam medidas rigorosas para proteger informações pessoais e financeiras, e a presença de um intermediário na plataforma ajuda a resolver disputas e conflitos entre compradores e vendedores, criando um ambiente de compra mais seguro e confiável. Em resumo, a adoção de sistemas de marketplace é fundamental para comércios acompanharem e se ajustarem a transformação do cenário comercial moderno.


Lauro César Araujo

% ----------------------------------------------------------
% PARTE
% ----------------------------------------------------------
\part{Definição do projeto}
% ----------------------------------------------------------

% ---
% Capitulo com exemplos de comandos inseridos de arquivo externo 
% ---
\include{abntex2-modelo-include-comandos}
% ---

\chapter{Plano de Negócios}\label{cap_plano_de_negocios}

\section{Desenvolvimento do Plano de Negócios}

\subsection{Resumo Executivo}

O "XY Sales" é um marketplace inovador que visa revolucionar a forma como as pessoas compram e vendem produtos online. Queremos conectar compradores e vendedores de maneira conveniente, eficaz e segura. Estamos focados em alcançar um crescimento sólido, atingindo um grande público e proporcionando uma experiência de compra excepcional.

% Insira a figura da logo aqui
\begin{figure}[htb]
	\centering
	\includegraphics[width=0.6\textwidth]{img/XY-logo}
	\caption{Logo desenvolvida para ilustrar a XY Sales (fonte: os autores).}
	\label{fig:logo-geekstore}
\end{figure}

\subsection{Descrição da Empresa}

O "XY Sales" foi fundado por um grupo de empreendedores apaixonados por comércio eletrônico. Nossa empresa é impulsionada por valores fundamentais de transparência, integridade e inovação, que estão no cerne de tudo o que fazemos.

\textbf{Visão:} Na "XY Sales", nossa visão é liderar a revolução do comércio eletrônico, conectando pessoas e negócios em uma comunidade global de prosperidade digital.

\textbf{Missão:} Nossa missão é proporcionar soluções de comércio eletrônico inovadoras, acessíveis e confiáveis, capacitando empreendedores e melhorando a vida dos consumidores em todo o mundo.

\textbf{Valores:} Somos guiados pela transparência em nossas ações, pela integridade em nossos relacionamentos e pela inovação contínua em tudo o que fazemos. Acreditamos no poder do comércio eletrônico para transformar vidas e estamos comprometidos em criar um impacto positivo, promovendo o crescimento econômico e a inclusão global.


\subsection{Análise de Mercado}

O mercado de comércio C2C (client to client) está em crescimento constante, com a demanda por compras online aumentando ano após ano e a necessidade - muitas vezes dificultada pela burocratização dos sistemas concorrentes - de vender o que já se tem.

Nossa análise de mercado identificou vários nichos específicos nos quais vemos oportunidades significativas de crescimento: 

\begin{itemize}
    \item \textbf{Escritórios de Contabilidade:} Escritórios de contabilidade frequentemente precisam atualizar seus equipamentos e tecnologia. Podemos fornecer soluções de tecnologia e mobiliário de escritório para ajudar esses escritórios a manterem-se atualizados e eficientes sem gastar muito dinheiro.

    \item \textbf{Setor de Saúde:} Clínicas médicas e dentárias podem se beneficiar de equipamentos médicos de alta qualidade e móveis específicos para a área da saúde, como cadeiras reclináveis ou macas.

    \item \textbf{Setor Hoteleiro:} Hotéis e pousadas frequentemente atualizam seus quartos e áreas comuns. Fornecemos móveis e eletrônicos de alta qualidade para atender às necessidades de renovação do setor hoteleiro.

    \item \textbf{Empresas de Tecnologia:} Empresas de tecnologia estão sempre em busca de equipamentos e dispositivos de última geração. Podemos ser um portal de fornecedores confiáveis para atender às necessidades de hardware e eletrônicos dessas empresas.
\end{itemize}

Embora acreditemos que nossa abordagem centrada no cliente nos diferenciará, em nossa pesquisa encontramos grandes empresas rivais à nossa:

\begin{itemize}
	\item \textbf{TechConnect Marketplace:} A TechConnect é especializada em eletrônicos e tecnologia. Eles oferecem uma ampla gama de dispositivos, gadgets e acessórios de alta tecnologia. Sua reputação se baseia na qualidade e inovação de seus produtos.

	\item \textbf{FashionHub Marketplace:} A FashionHub é uma plataforma de moda e vestuário que atende a consumidores e vendedores de moda em todo o mundo. Eles estão na vanguarda das últimas tendências da moda e oferecem uma experiência de compra exclusiva.

	\item \textbf{HomeRenovate Marketplace:} A HomeRenovate é um mercado especializado em produtos para reformas e decoração de interiores. Eles se destacam por oferecer uma variedade abrangente de materiais de construção, móveis e decoração para ajudar os clientes a transformar suas casas.

\end{itemize}

\subsection{Estratégia e Plano de Marketing}

Para nos destacarmos, iremos enfatizar a qualidade do serviço ao cliente, oferecer uma ampla variedade de produtos, implementar um sistema de avaliação de vendedores confiável e utilizar estratégias de marketing online, como anúncios direcionados nas redes sociais e otimização de mecanismos de busca (SEO).

\subsection{Plano Operacional}

Nossa operação contará com uma plataforma online de fácil utilização para os dois perfís de clientes: quem busca comprar e quem busca vender. Também contaremos com uma rede de parceiros de envio para lidar com a logística e uma estrutura de atendimento ao cliente eficiente e rápida, com uma integração parcial à Inteligência Artificial (IA). Manteremos estoque mínimo conosco - os produtos comercializados serão buscados na casa dos vendedores e serão transportados diretamente ao cliente final.

\subsection{Plano Financeiro}

Projetamos receitas crescentes com base em taxas de comissão de vendas e publicidade. Prevemos um aumento constante no número de vendedores e compradores, com um fluxo de caixa estável no primeiro ano, onde ainda estaremos pagando o investimento inicial, no segundo ano prevemos ganho ascendente de no mínimo 8\% com metade dos custos. Estamos prontos para lidar com diferentes cenários de crescimento e queda.

\subsection{Gestão e Equipe}

Nossa equipe de gestão será composta por pessoal competente selecionado a dedo por processos eliminatórios, serão propostos cargos em nível de senioridade em todos as áreas, trazendo experiência em comércio eletrônico. Os cargos principais que buscamos serão: Desenvolvimento, Recursos Humanos, Gerente de Projetos, UX-UI Designers, Marketing, Sales Force e Designers.

\subsection{Estrutura de Propriedade e Jurídica}

A propriedade do "XY Sales" é distribuída entre os fundadores, com 35\%  da empresa cotada para participação acionária inicialmente, um valor aparentemente baixo, já que o investimento inicial foi baixo. Estamos em conformidade com todas as regulamentações do setor e garantimos a proteção da propriedade intelectual de nossas ideias.

\subsection{Plano de Implementação}

Nosso plano de implementação inclui o lançamento da plataforma no primeiro trimestre, nesse estágio faremos parceria com outras entidades: parceiros e influenciadores digitais, estes fomentarão o lucro do trimestre permitindo que continuemos as nossas atividades.

No segundo trimestre prevemos uma expansão de vendedores e compradores por meio das ações do marketing. Estabelecemos metas específicas, como alcançar no mínimo 1000 vendedores e 5000 vendas até o final do ano - um valor de lucro que estimamos em 5.000.000 em reais.

\section{Requisitos Funcionais e Não Funcionais}

Para o desenvolvimento do projeto da plataforma, foram idealizados os seguintes Requisitos Não Funcionais, presentes no \autoref{quadro_rnf} . 

Também foram analizados os Requisitos Funcionais, presentes no \autoref{quadro_rf} .

\begin{quadro}[htb]
	\caption{\label{quadro_rnf}Requisitos Não Funcionais}
	\begin{adjustbox}{center}
		\begin{tabular}{|p{2cm}|p{4cm}|p{4cm}|p{2cm}|p{5cm}|}
			\hline
			\textbf{Id} & \textbf{Nome} & \textbf{Categoria} & \textbf{Prioridade} & \textbf{Descrição} \\ \hline
			[RNF001] & Portabilidade de telas & Compatibilidade & Essencial & Interface deverá se adaptar às telas de diferentes dispositivos visto que no mundo atual há uma ampla variedade de dispositivos pelos quais pode-se acessar um marketplace. \\ \hline
			[RNF002] & Interface intuitiva & Usabilidade & Essencial & A interface deve ser de fácil interpretação para os usuários a fim de que eles consigam localizar e acessar as funções desejadas em poucos segundos. \\ \hline
			[RNF003] & Boa performance & Desempenho & Essencial & Cliente deverá obter respostas rápidas para suas pesquisas e páginas deverão ser carregadas com agilidade. \\ \hline
			[RNF004] & Backup de dados & Disponibilidade & Essencial & Sistema registrará e arquivará todas as informações relevantes de seus usuários para uso futuro. \\ \hline
			[RNF005] & Segurança de acesso & Segurança & Essencial & Cada usuário terá login e senha próprio para acessar sua conta. \\ \hline
			[RNF006] & Facilidade de manutenção & Padrões & Essencial & Serão usados métodos estruturados de código para facilitar sua manutenção e reusabilidade futuramente. \\ \hline
			[RNF007] & Estrutura do código & Padrões & Essencial & Código será desenvolvido na linguagem C\#. \\ \hline
			[RNF008] & Controle de compras & Segurança & Essencial & O sistema designará um código específico para cada compra a fim de localizá-la. \\ \hline
		\end{tabular}
	\end{adjustbox}
\fonte{fonte: os autores}
\end{quadro}

\begin{quadro}[htb]
	\caption{\label{quadro_rf}Requisitos Funcionais}
	\begin{adjustbox}{center}
		\begin{tabular}{|p{2cm}|p{4cm}|p{4cm}|p{2cm}|p{5cm}|}
			\hline
			\textbf{Id} & \textbf{Função} & \textbf{Prioridade} & \textbf{Ator} & \textbf{Descrição} \\ \hline
			[RF001] & PESQUISA & Essencial & Sistema & O sistema deverá permitir o usuário pesquisar produto. \\ \hline
			[RF002] & AVALIAÇÃO & Essencial & Sistema & O sistema deverá permitir o usuário ler avaliação/avaliar. \\ \hline
			[RF003] & ESCOLHA & Essencial & Sistema & O sistema deverá permitir o usuário escolher atributos do produto. \\ \hline
			[RF004] & ADICIONAR & Essencial & Sistema & O sistema deverá permitir o usuário adicionar ao carrinho. \\ \hline
			[RF005] & VISUALIZAR & Essencial & Sistema & O sistema deverá permitir o usuário ver carrinho. \\ \hline
			[RF006] & FINALIZAR & Essencial & Sistema & O sistema deverá permitir o usuário finalizar compra. \\ \hline
			[RF007] & ENTREGA & Essencial & Sistema & O sistema deverá permitir o usuário configurar opções de entrega. \\ \hline
			[RF008] & PAGAMENTO & Essencial & Sistema & O sistema deverá permitir o usuário configurar opções de pagamento. \\ \hline
			[RF009] & CADASTRO & Essencial & Sistema & O sistema deverá permitir o usuário configurar opções de cadastro. \\ \hline
			[RF010] & CHAT & Essencial & Sistema & O sistema deverá permitir o usuário conversar em tempo real. \\ \hline
			[RF011] & CONSULTA & Essencial & Sistema & O sistema deverá permitir o usuário consultar dados. \\ \hline
			[RF012] & CARRINHO & Essencial & Sistema & O sistema deverá permitir o usuário configurar opções do carrinho. \\ \hline
			[RF013] & FRETE & Essencial & Sistema & O sistema deverá permitir o usuário calcular frete. \\ \hline
		\end{tabular}
	\end{adjustbox}
\fonte{fonte: os autores}
\end{quadro}

% ----------------------------------------------------------
% PARTE
% ----------------------------------------------------------
\part{Referenciais teóricos}
% ----------------------------------------------------------

% ---
% Capitulo de revisão de literatura
% ---
\chapter{Lorem ipsum dolor sit amet}
% ---

% ---
\section{Aliquam vestibulum fringilla lorem}
% ---

\lipsum[1]

\lipsum[2-3]

% ----------------------------------------------------------
% PARTE
% ----------------------------------------------------------
\part{Resultados}
% ----------------------------------------------------------

% ---
% primeiro capitulo de Resultados
% ---
\chapter{Lectus lobortis condimentum}
% ---

% ---
\section{Vestibulum ante ipsum primis in faucibus orci luctus et ultrices
posuere cubilia Curae}
% ---

\lipsum[21-22]

% ---
% segundo capitulo de Resultados
% ---
\chapter{Nam sed tellus sit amet lectus urna ullamcorper tristique interdum
elementum}
% ---

% ---
\section{Pellentesque sit amet pede ac sem eleifend consectetuer}
% ---

\lipsum[24]

% ----------------------------------------------------------
% Finaliza a parte no bookmark do PDF
% para que se inicie o bookmark na raiz
% e adiciona espaço de parte no Sumário
% ----------------------------------------------------------
\phantompart

% ---
% Conclusão
% ---
\chapter{Conclusão}
% ---

\lipsum[31-33]

% ----------------------------------------------------------
% ELEMENTOS PÓS-TEXTUAIS
% ----------------------------------------------------------
\postextual
% ----------------------------------------------------------

% ----------------------------------------------------------
% Referências bibliográficas
% ----------------------------------------------------------
\bibliography{abntex2-modelo-references}

% ----------------------------------------------------------
% Glossário
% ----------------------------------------------------------
%
% Consulte o manual da classe abntex2 para orientações sobre o glossário.
%
%\glossary

% ----------------------------------------------------------
% Apêndices
% ----------------------------------------------------------

% ---
% Inicia os apêndices
% ---
\begin{apendicesenv}

% Imprime uma página indicando o início dos apêndices
\partapendices

% ----------------------------------------------------------
\chapter{Quisque libero justo}
% ----------------------------------------------------------

\lipsum[50]

% ----------------------------------------------------------
\chapter{Nullam elementum urna vel imperdiet sodales elit ipsum pharetra ligula
ac pretium ante justo a nulla curabitur tristique arcu eu metus}
% ----------------------------------------------------------
\lipsum[55-57]

\end{apendicesenv}
% ---


% ----------------------------------------------------------
% Anexos
% ----------------------------------------------------------

% ---
% Inicia os anexos
% ---
\begin{anexosenv}

% Imprime uma página indicando o início dos anexos
\partanexos

% ---
\chapter{Morbi ultrices rutrum lorem.}
% ---
\lipsum[30]

% ---
\chapter{Cras non urna sed feugiat cum sociis natoque penatibus et magnis dis
parturient montes nascetur ridiculus mus}
% ---

\lipsum[31]

% ---
\chapter{Fusce facilisis lacinia dui}
% ---

\lipsum[32]

\end{anexosenv}

%---------------------------------------------------------------------
% INDICE REMISSIVO
%---------------------------------------------------------------------
\phantompart
\printindex
%---------------------------------------------------------------------

\end{document}
